\documentclass[aspectratio=169]{beamer}
\usepackage{graphicx}
\usepackage{booktabs}
\usepackage{hyperref}

% Theme
\usetheme{Madrid}
\usecolortheme{default}

% Title page
\title{Nghiên cứu và Thực nghiệm Hệ thống RAG}
\subtitle{Retrieval-Augmented Generation: Best Practices}
\author{[Họ và tên sinh viên]}
\institute{
    [Tên trường] \\
    \medskip
    Giảng viên hướng dẫn: [Tên giảng viên]
}
\date{\today}

\begin{document}

%=================================================================
% SLIDE 1: Title
%=================================================================
\frame{\titlepage}

%=================================================================
% SLIDE 2: Mục lục
%=================================================================
\begin{frame}{Nội dung trình bày}
    \tableofcontents
\end{frame}

%=================================================================
\section{Giới thiệu}
%=================================================================

%=================================================================
% SLIDE 3: RAG là gì?
%=================================================================
\begin{frame}{RAG là gì?}
    \begin{block}{Retrieval-Augmented Generation (RAG)}
        Kỹ thuật kết hợp tra cứu thông tin và tạo văn bản tự động
    \end{block}
    
    \vspace{0.5cm}
    
    \begin{columns}
        \begin{column}{0.5\textwidth}
            \textbf{Quy trình:}
            \begin{enumerate}
                \item Nhận câu hỏi
                \item Tìm kiếm tài liệu liên quan
                \item Trích xuất thông tin quan trọng
                \item Tạo câu trả lời chính xác
            \end{enumerate}
        \end{column}
        
        \begin{column}{0.5\textwidth}
            \begin{alertblock}{Ví dụ}
                Giống như làm bài tập được tra sách giáo khoa thay vì chỉ nhớ bài!
            \end{alertblock}
        \end{column}
    \end{columns}
\end{frame}

%=================================================================
% SLIDE 4: Tại sao nghiên cứu RAG?
%=================================================================
\begin{frame}{Tại sao nghiên cứu RAG?}
    \begin{block}{Vấn đề của LLM hiện tại}
        \begin{itemize}
            \item \textbf{Kiến thức tĩnh:} Không cập nhật thông tin mới
            \item \textbf{Hallucination:} Có xu hướng "bịa đặt" thông tin
            \item \textbf{Thiếu minh bạch:} Khó kiểm tra nguồn gốc
        \end{itemize}
    \end{block}
    
    \vspace{0.5cm}
    
    \begin{exampleblock}{Giải pháp RAG}
        \begin{itemize}
            \item Tra cứu từ cơ sở tri thức bên ngoài
            \item Cung cấp nguồn tham khảo rõ ràng
            \item Có thể cập nhật kiến thức dễ dàng
        \end{itemize}
    \end{exampleblock}
\end{frame}

%=================================================================
\section{Kiến trúc hệ thống}
%=================================================================

%=================================================================
% SLIDE 5: Kiến trúc tổng quan
%=================================================================
\begin{frame}{Kiến trúc hệ thống RAG}
    \begin{figure}
        \centering
        \includegraphics[width=0.85\textwidth]{rag-diagram.png}
        \caption{Sơ đồ kiến trúc RAG}
    \end{figure}
\end{frame}

%=================================================================
% SLIDE 6: Ba module chính
%=================================================================
\begin{frame}{Ba Module chính}
    \begin{enumerate}
        \item \textbf{Query Expansion Module}
        \begin{itemize}
            \item Mô hình: Flan-T5-small
            \item Mở rộng câu hỏi thành các từ khóa liên quan
        \end{itemize}
        
        \vspace{0.3cm}
        
        \item \textbf{Retrieval Module}
        \begin{itemize}
            \item Embedding: all-MiniLM-L6-v2
            \item Vector DB: FAISS
            \item Chunk size: 64 tokens, overlap: 8 tokens
        \end{itemize}
        
        \vspace{0.3cm}
        
        \item \textbf{Generation Module}
        \begin{itemize}
            \item Mô hình: Mistral-7B-Instruct-v0.2
            \item Sinh câu trả lời dựa trên ngữ cảnh
        \end{itemize}
    \end{enumerate}
\end{frame}

%=================================================================
\section{Các phương pháp cải tiến}
%=================================================================

%=================================================================
% SLIDE 7: Query Expansion
%=================================================================
\begin{frame}{Phương pháp 1: Query Expansion}
    \begin{block}{Ý tưởng}
        Mở rộng câu hỏi bằng cách thêm từ khóa liên quan
    \end{block}
    
    \vspace{0.5cm}
    
    \begin{columns}
        \begin{column}{0.5\textwidth}
            \textbf{Ví dụ:}
            \begin{itemize}
                \item \textcolor{blue}{Gốc:} "Python là gì?"
                \item \textcolor{green}{Mở rộng:} "Python là gì + ngôn ngữ lập trình + ứng dụng Python"
            \end{itemize}
        \end{column}
        
        \begin{column}{0.5\textwidth}
            \textbf{Lợi ích:}
            \begin{itemize}
                \item Tăng diện tích tìm kiếm
                \item Tìm được nhiều tài liệu liên quan
                \item Đặc biệt hiệu quả với câu hỏi ngắn
            \end{itemize}
        \end{column}
    \end{columns}
\end{frame}

%=================================================================
% SLIDE 8: Focus Mode
%=================================================================
\begin{frame}{Phương pháp 2: Focus Mode}
    \begin{block}{Ý tưởng}
        Chỉ trích xuất những câu quan trọng nhất thay vì cả đoạn văn
    \end{block}
    
    \vspace{0.5cm}
    
    \textbf{Cách hoạt động:}
    \begin{enumerate}
        \item Tìm kiếm ở cấp độ đoạn văn
        \item Tìm kiếm tiếp ở cấp độ câu
        \item Chỉ giữ lại những câu có độ tương đồng cao nhất
    \end{enumerate}
    
    \vspace{0.3cm}
    
    \begin{exampleblock}{Ví dụ}
        Giống như đọc sách chỉ highlight những dòng quan trọng nhất!
    \end{exampleblock}
    
    \vspace{0.3cm}
    
    \textbf{Lợi ích:} Giảm nhiễu, tăng độ chính xác
\end{frame}

%=================================================================
% SLIDE 9: In-Context Learning
%=================================================================
\begin{frame}{Phương pháp 3: Contrastive ICL}
    \begin{block}{Ý tưởng}
        Dạy mô hình phân biệt đúng-sai bằng cách đưa ví dụ trong prompt
    \end{block}
    
    \vspace{0.3cm}
    
    \begin{columns}
        \begin{column}{0.5\textwidth}
            \textbf{Ví dụ đúng:}
            \begin{itemize}
                \item Q: Thủ đô Việt Nam?
                \item Context: \textcolor{green}{Hà Nội là thủ đô...}
                \item A: Hà Nội
            \end{itemize}
        \end{column}
        
        \begin{column}{0.5\textwidth}
            \textbf{Ví dụ sai:}
            \begin{itemize}
                \item Q: Thủ đô Việt Nam?
                \item Context: \textcolor{red}{Tokyo là thủ đô...}
                \item A: Không thể trả lời
            \end{itemize}
        \end{column}
    \end{columns}
    
    \vspace{0.5cm}
    
    \begin{alertblock}{Kết quả}
        Mô hình học cách đánh giá độ tin cậy của thông tin
    \end{alertblock}
\end{frame}

%=================================================================
\section{Dữ liệu và Cấu hình}
%=================================================================

%=================================================================
% SLIDE 10: Bộ dữ liệu
%=================================================================
\begin{frame}{Bộ dữ liệu sử dụng}
    \begin{columns}
        \begin{column}{0.5\textwidth}
            \begin{block}{Dữ liệu Đánh giá}
                \textbf{1. TruthfulQA}
                \begin{itemize}
                    \item 817 câu hỏi
                    \item Kiểm tra tính trung thực
                    \item Nhiều câu "bẫy"
                \end{itemize}
                
                \vspace{0.3cm}
                
                \textbf{2. MMLU}
                \begin{itemize}
                    \item Đa lĩnh vực
                    \item Kiến thức chuyên môn
                    \item Từ cơ bản đến chuyên gia
                \end{itemize}
            \end{block}
        \end{column}
        
        \begin{column}{0.5\textwidth}
            \begin{block}{Cơ sở Tri thức}
                \textbf{Wikipedia Vital Articles}
                \begin{itemize}
                    \item Level 3 \& 4
                    \item Hàng nghìn bài viết chất lượng
                    \item Bao phủ nhiều chủ đề
                \end{itemize}
                
                \vspace{0.3cm}
                
                \textbf{Xử lý:}
                \begin{itemize}
                    \item Chunk: 64 tokens
                    \item Overlap: 8 tokens
                    \item Vector index: FAISS
                \end{itemize}
            \end{block}
        \end{column}
    \end{columns}
\end{frame}

%=================================================================
% SLIDE 11: Cấu hình thực nghiệm
%=================================================================
\begin{frame}{7 Cấu hình thực nghiệm}
    \begin{table}
        \centering
        \small
        \begin{tabular}{@{}clccc@{}}
            \toprule
            \textbf{\#} & \textbf{Tên} & \textbf{Expand} & \textbf{Focus} & \textbf{ICL} \\ 
            \midrule
            1 & Baseline & ✗ & ✗ & ✗ \\
            2 & ExpandQuery Only & ✓ & ✗ & ✗ \\
            3 & Focus Only & ✗ & ✓ & ✗ \\
            4 & ICL Only & ✗ & ✗ & ✓ \\
            5 & ExpandQuery + Focus & ✓ & ✓ & ✗ \\
            6 & Focus + ICL & ✗ & ✓ & ✓ \\
            7 & Hybrid All Features & ✓ & ✓ & ✓ \\
            \bottomrule
        \end{tabular}
    \end{table}
    
    \vspace{0.5cm}
    
    \textbf{Chỉ số đánh giá:}
    \begin{itemize}
        \item ROUGE (R1, R2, RL) - Độ trùng lặp n-gram
        \item Similarity - Độ tương đồng ngữ nghĩa
        \item MAUVE - Chất lượng văn bản sinh ra
    \end{itemize}
\end{frame}

%=================================================================
\section{Kết quả thực nghiệm}
%=================================================================

%=================================================================
% SLIDE 12: Kết quả MMLU
%=================================================================
\begin{frame}{Kết quả trên MMLU}
    \begin{table}
        \centering
        \tiny
        \begin{tabular}{@{}lccccc@{}}
            \toprule
            \textbf{Cấu hình} & \textbf{R1-F1} & \textbf{R2-F1} & \textbf{RL-F1} & \textbf{Similarity} & \textbf{MAUVE} \\ 
            \midrule
            1. Baseline & 0.1021 & 0.0189 & 0.0870 & 0.2881 & 0.6502 \\
            2. ExpandQuery Only & 0.1040 & 0.0211 & 0.0893 & 0.2926 & 0.6210 \\
            3. Focus Only & 0.1010 & 0.0201 & 0.0870 & 0.2913 & 0.3636 \\
            4. ICL Only & \textbf{0.1068} & \textbf{0.0239} & \textbf{0.0921} & 0.3044 & 0.5407 \\
            5. ExpandQuery+Focus & 0.1013 & 0.0196 & 0.0865 & 0.2890 & 0.6028 \\
            6. Focus+ICL & \textcolor{red}{\textbf{0.1130}} & \textcolor{red}{\textbf{0.0242}} & \textcolor{red}{\textbf{0.0966}} & \textcolor{red}{\textbf{0.3141}} & 0.4058 \\
            7. Hybrid All & 0.0973 & 0.0210 & 0.0843 & 0.2869 & 0.4884 \\
            \bottomrule
        \end{tabular}
    \end{table}
    
    \vspace{0.5cm}
    
    \begin{block}{Nhận xét}
        \begin{itemize}
            \item \textcolor{red}{\textbf{Config 6 (Focus+ICL)}} đạt kết quả tốt nhất (+10.7\% R1-F1)
            \item ICL Only (Config 4) cũng tốt và \textbf{nhanh nhất}
            \item Hybrid All không tốt như mong đợi trên MMLU
        \end{itemize}
    \end{block}
\end{frame}

%=================================================================
% SLIDE 13: Kết quả TruthfulQA
%=================================================================
\begin{frame}{Kết quả trên TruthfulQA}
    \begin{table}
        \centering
        \tiny
        \begin{tabular}{@{}lccccc@{}}
            \toprule
            \textbf{Cấu hình} & \textbf{R1-F1} & \textbf{R2-F1} & \textbf{RL-F1} & \textbf{Similarity} & \textbf{MAUVE} \\ 
            \midrule
            1. Baseline & 0.2562 & 0.1195 & 0.2265 & 0.5589 & 0.4933 \\
            2. ExpandQuery Only & 0.2604 & 0.1208 & 0.2286 & 0.5667 & 0.4726 \\
            3. Focus Only & 0.2631 & 0.1236 & 0.2320 & 0.5663 & 0.5013 \\
            4. ICL Only & 0.2726 & 0.1444 & 0.2428 & 0.5379 & 0.2959 \\
            6. Focus+ICL & 0.2745 & 0.1306 & 0.2422 & 0.5795 & 0.4187 \\
            7. Hybrid All & \textcolor{red}{\textbf{0.3823}} & \textcolor{red}{\textbf{0.2755}} & \textcolor{red}{\textbf{0.3611}} & \textcolor{red}{\textbf{0.6332}} & 0.4604 \\
            \bottomrule
        \end{tabular}
    \end{table}
    
    \vspace{0.5cm}
    
    \begin{block}{Nhận xét}
        \begin{itemize}
            \item \textcolor{red}{\textbf{Config 7 (Hybrid All)}} vượt trội: +49\% R1-F1, +131\% R2-F1!
            \item Kết hợp cả 3 phương pháp rất hiệu quả cho TruthfulQA
            \item Focus+ICL (Config 6) cũng cho kết quả tốt
        \end{itemize}
    \end{block}
\end{frame}

%=================================================================
% SLIDE 14: Biểu đồ ROUGE
%=================================================================
\begin{frame}{So sánh ROUGE Metrics}
    \begin{figure}
        \centering
        \includegraphics[width=0.95\textwidth]{rouge_metrics_comparison.png}
    \end{figure}
\end{frame}

%=================================================================
% SLIDE 15: Biểu đồ Similarity & MAUVE
%=================================================================
\begin{frame}{So sánh Similarity và MAUVE}
    \begin{figure}
        \centering
        \includegraphics[width=0.95\textwidth]{similarity_mauve_comparison.png}
    \end{figure}
\end{frame}

%=================================================================
% SLIDE 16: Heatmap
%=================================================================
\begin{frame}{Heatmap: Tổng quan các chỉ số}
    \begin{figure}
        \centering
        \includegraphics[width=0.90\textwidth]{metrics_heatmap.png}
    \end{figure}
\end{frame}

%=================================================================
% SLIDE 17: MMLU Dashboard
%=================================================================
\begin{frame}{MMLU - Dashboard tổng hợp}
    \begin{figure}
        \centering
        \includegraphics[width=0.85\textwidth]{mmlu_comprehensive_dashboard.png}
    \end{figure}
\end{frame}

%=================================================================
% SLIDE 18: TruthfulQA Dashboard
%=================================================================
\begin{frame}{TruthfulQA - Dashboard tổng hợp}
    \begin{figure}
        \centering
        \includegraphics[width=0.85\textwidth]{truthfulqa_comprehensive_dashboard.png}
    \end{figure}
\end{frame}

%=================================================================
% SLIDE 19: Improvement Analysis
%=================================================================
\begin{frame}{Phân tích cải thiện so với Baseline}
    \begin{figure}
        \centering
        \includegraphics[width=0.95\textwidth]{improvement_analysis.png}
    \end{figure}
\end{frame}

%=================================================================
% SLIDE 20: Timing Analysis
%=================================================================
\begin{frame}{Phân tích thời gian thực thi}
    \begin{figure}
        \centering
        \includegraphics[width=0.95\textwidth]{timing_analysis.png}
    \end{figure}
\end{frame}

%=================================================================
% SLIDE 21: Timing Table
%=================================================================
\begin{frame}{So sánh thời gian thực thi}
    \begin{table}
        \centering
        \tiny
        \begin{tabular}{@{}lcccc@{}}
            \toprule
            \textbf{Cấu hình} & \textbf{Model Load} & \textbf{RAG Init} & \textbf{Evaluation} & \textbf{Total} \\ 
            \midrule
            \multicolumn{5}{c}{\textit{MMLU}} \\
            \midrule
            1. Baseline & 77.3s & 185.1s & 4979.5s & 5244.1s \\
            4. ICL Only & 41.1s & \textcolor{green}{\textbf{4.8s}} & \textcolor{green}{\textbf{3412.4s}} & \textcolor{green}{\textbf{3459.4s}} \\
            6. Focus+ICL & 33.0s & 4.0s & 3762.1s & 3800.2s \\
            7. Hybrid All & 35.8s & 188.0s & 6353.0s & \textcolor{red}{6579.2s} \\
            \midrule
            \multicolumn{5}{c}{\textit{TruthfulQA}} \\
            \midrule
            1. Baseline & 65.3s & 160.1s & 1454.8s & 1682.1s \\
            4. ICL Only & 62.7s & \textcolor{green}{\textbf{7.3s}} & \textcolor{green}{\textbf{1326.8s}} & \textcolor{green}{\textbf{1397.6s}} \\
            6. Focus+ICL & 38.4s & 7.3s & 1454.0s & 1500.6s \\
            7. Hybrid All & 25.6s & 158.3s & 1769.5s & 1955.5s \\
            \bottomrule
        \end{tabular}
    \end{table}
    
    \begin{block}{Nhận xét}
        \textbf{ICL} giảm thời gian đáng kể (34-40\% nhanh hơn Baseline)
    \end{block}
\end{frame}

%=================================================================
\section{Kết luận}
%=================================================================

%=================================================================
% SLIDE 22: Những phát hiện chính
%=================================================================
\begin{frame}{Những phát hiện chính}
    \begin{enumerate}
        \item \textbf{Contrastive ICL hiệu quả nhất}
        \begin{itemize}
            \item Dạy mô hình phân biệt đúng-sai rất quan trọng
            \item Tăng cả độ chính xác và tốc độ
        \end{itemize}
        
        \vspace{0.3cm}
        
        \item \textbf{Focus Mode quan trọng}
        \begin{itemize}
            \item Chất lượng ngữ cảnh > Số lượng
            \item Lọc thông tin giúp mô hình tập trung tốt hơn
        \end{itemize}
        
        \vspace{0.3cm}
        
        \item \textbf{Kết hợp phương pháp hiệu quả}
        \begin{itemize}
            \item Hybrid All tốt nhất trên TruthfulQA
            \item Focus+ICL cân bằng tốc độ và độ chính xác
        \end{itemize}
        
        \vspace{0.3cm}
        
        \item \textbf{Trade-off tốc độ vs độ chính xác}
        \begin{itemize}
            \item ICL-only: nhanh nhưng độ chính xác trung bình
            \item Hybrid: chậm hơn nhưng chính xác nhất
        \end{itemize}
    \end{enumerate}
\end{frame}

%=================================================================
% SLIDE 23: Ứng dụng thực tế
%=================================================================
\begin{frame}{Ứng dụng thực tế}
    \begin{columns}
        \begin{column}{0.5\textwidth}
            \textbf{Lĩnh vực áp dụng:}
            \begin{itemize}
                \item Chatbot, trợ lý ảo
                \item Hệ thống hỏi đáp tự động
                \item Công cụ tìm kiếm thông minh
                \item Giáo dục - Gia sư tự động
                \item Y tế, luật pháp, v.v.
            \end{itemize}
        \end{column}
        
        \begin{column}{0.5\textwidth}
            \textbf{Hướng phát triển:}
            \begin{itemize}
                \item Thử mô hình lớn hơn
                \item Tối ưu tham số
                \item Multilingual RAG
                \item Domain-specific RAG
                \item Real-time update
            \end{itemize}
        \end{column}
    \end{columns}
    
    \vspace{0.8cm}
    
    \begin{exampleblock}{Lời khuyên thực tế}
        \begin{itemize}
            \item Dùng Focus+ICL cho ứng dụng production (cân bằng)
            \item Dùng ICL-only nếu cần tốc độ
            \item Dùng Hybrid All nếu cần độ chính xác cao nhất
        \end{itemize}
    \end{exampleblock}
\end{frame}

%=================================================================
% SLIDE 24: Q&A
%=================================================================
\begin{frame}
    \Huge
    \textbf{Cảm ơn!}
    
    \vspace{1cm}
    
    \Large
    Câu hỏi và thảo luận
    
    \vspace{1cm}
    
    \normalsize
    \textbf{GitHub:} \url{https://github.com/ali-bahrainian/RAG_best_practices}
\end{frame}

%=================================================================
% SLIDE 25: Backup - Nguồn tham khảo
%=================================================================
\begin{frame}{Nguồn tham khảo}
    \begin{enumerate}
        \small
        \item Li, S., Stenzel, L., Eickhoff, C., \& Bahrainian, S. A. (2025). \textit{Enhancing Retrieval-Augmented Generation: A Study of Best Practices}. arXiv:2501.07391.
        
        \item Lewis, P., et al. (2020). \textit{Retrieval-Augmented Generation for Knowledge-Intensive NLP Tasks}. NeurIPS 2020.
        
        \item GitHub: \url{https://github.com/ali-bahrainian/RAG_best_practices}
        
        \item Hugging Face Transformers
        
        \item FAISS Documentation
        
        \item TruthfulQA \& MMLU Datasets
    \end{enumerate}
\end{frame}

\end{document}
