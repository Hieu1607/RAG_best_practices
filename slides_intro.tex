%=================================================================
\section{Giới thiệu}
%=================================================================

%=================================================================
% SLIDE 3: RAG là gì?
%=================================================================
\begin{frame}{RAG là gì?}
    \begin{block}{Retrieval-Augmented Generation (RAG)}
        Kỹ thuật kết hợp tra cứu thông tin và tạo văn bản tự động
    \end{block}
    
    \vspace{0.5cm}
    
    \begin{columns}
        \begin{column}{0.5\textwidth}
            \textbf{Quy trình:}
            \begin{enumerate}
                \item Nhận câu hỏi
                \item Tìm kiếm tài liệu liên quan
                \item Trích xuất thông tin quan trọng
                \item Tạo câu trả lời chính xác
            \end{enumerate}
        \end{column}
        
        \begin{column}{0.5\textwidth}
            \begin{alertblock}{Ví dụ}
                Giống như làm bài tập được tra sách giáo khoa thay vì chỉ nhớ bài!
            \end{alertblock}
        \end{column}
    \end{columns}
\end{frame}

%=================================================================
% SLIDE 4: Tại sao nghiên cứu RAG?
%=================================================================
\begin{frame}{Tại sao nghiên cứu RAG?}
    \begin{block}{Vấn đề của LLM hiện tại}
        \begin{itemize}
            \item \textbf{Kiến thức tĩnh:} Không cập nhật thông tin mới
            \item \textbf{Hallucination:} Có xu hướng "bịa đặt" thông tin
            \item \textbf{Thiếu minh bạch:} Khó kiểm tra nguồn gốc
        \end{itemize}
    \end{block}
    
    \vspace{0.5cm}
    
    \begin{exampleblock}{Giải pháp RAG}
        \begin{itemize}
            \item Tra cứu từ cơ sở tri thức bên ngoài
            \item Cung cấp nguồn tham khảo rõ ràng
            \item Có thể cập nhật kiến thức dễ dàng
        \end{itemize}
    \end{exampleblock}
\end{frame}
