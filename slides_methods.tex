%=================================================================
\section{Các phương pháp cải tiến}
%=================================================================

%=================================================================
% SLIDE 7: Query Expansion
%=================================================================
\begin{frame}{Phương pháp 1: Query Expansion}
    \begin{block}{Ý tưởng}
        Mở rộng câu hỏi bằng cách thêm từ khóa liên quan
    \end{block}
    
    \vspace{0.5cm}
    
    \begin{columns}
        \begin{column}{0.5\textwidth}
            \textbf{Ví dụ:}
            \begin{itemize}
                \item \textcolor{blue}{Gốc:} "Python là gì?"
                \item \textcolor{green}{Mở rộng:} "Python là gì + ngôn ngữ lập trình + ứng dụng Python"
            \end{itemize}
        \end{column}
        
        \begin{column}{0.5\textwidth}
            \textbf{Lợi ích:}
            \begin{itemize}
                \item Tăng diện tích tìm kiếm
                \item Tìm được nhiều tài liệu liên quan
                \item Đặc biệt hiệu quả với câu hỏi ngắn
            \end{itemize}
        \end{column}
    \end{columns}
\end{frame}

%=================================================================
% SLIDE 8: Focus Mode
%=================================================================
\begin{frame}{Phương pháp 2: Focus Mode}
    \begin{block}{Ý tưởng}
        Chỉ trích xuất những câu quan trọng nhất thay vì cả đoạn văn
    \end{block}
    
    \vspace{0.5cm}
    
    \textbf{Cách hoạt động:}
    \begin{enumerate}
        \item Tìm kiếm ở cấp độ đoạn văn
        \item Tìm kiếm tiếp ở cấp độ câu
        \item Chỉ giữ lại những câu có độ tương đồng cao nhất
    \end{enumerate}
    
    \vspace{0.3cm}
    
    \begin{exampleblock}{Ví dụ}
        Giống như đọc sách chỉ highlight những dòng quan trọng nhất!
    \end{exampleblock}
    
    \vspace{0.3cm}
    
    \textbf{Lợi ích:} Giảm nhiễu, tăng độ chính xác
\end{frame}

%=================================================================
% SLIDE 9: In-Context Learning
%=================================================================
\begin{frame}{Phương pháp 3: Contrastive ICL}
    \begin{block}{Ý tưởng}
        Dạy mô hình phân biệt đúng-sai bằng cách đưa ví dụ trong prompt
    \end{block}
    
    \vspace{0.3cm}
    
    \begin{columns}
        \begin{column}{0.5\textwidth}
            \textbf{Ví dụ đúng:}
            \begin{itemize}
                \item Q: Thủ đô Việt Nam?
                \item Context: \textcolor{green}{Hà Nội là thủ đô...}
                \item A: Hà Nội
            \end{itemize}
        \end{column}
        
        \begin{column}{0.5\textwidth}
            \textbf{Ví dụ sai:}
            \begin{itemize}
                \item Q: Thủ đô Việt Nam?
                \item Context: \textcolor{red}{Tokyo là thủ đô...}
                \item A: Không thể trả lời
            \end{itemize}
        \end{column}
    \end{columns}
    
    \vspace{0.5cm}
    
    \begin{alertblock}{Kết quả}
        Mô hình học cách đánh giá độ tin cậy của thông tin
    \end{alertblock}
\end{frame}
